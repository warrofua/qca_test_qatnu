% SRQID — arXiv Draft (Aug 13, 2025)
\documentclass[12pt,a4paper]{article}
\usepackage[utf8]{inputenc}
\usepackage[T1]{fontenc}
\usepackage{amsmath,amssymb,amsthm}
\usepackage{physics}
\usepackage{graphicx}
\usepackage{color}
\usepackage{float}
\usepackage{booktabs}
\usepackage{siunitx}
\usepackage{tikz}
\usepackage{pgfplots}
\usepackage{subcaption}
\usepackage{geometry}
\geometry{margin=1in}
\pgfplotsset{compat=1.18}

% theorem-like environments
\newtheorem{theorem}{Theorem}
\newtheorem{proposition}{Proposition}
\newtheorem{lemma}{Lemma}
\theoremstyle{remark}
\newtheorem*{remark}{Remark}

% Custom commands
\newcommand{\srqid}{\textsc{srqid}}
\newcommand{\hilbert}{\mathcal{H}}
\newcommand{\expect}[1]{\langle #1 \rangle}
% Avoid conflict with physics.sty which already defines \tr
\providecommand{\tr}{\mathrm{tr}}
\newcommand{\id}{\mathbb{I}}
\newcommand{\Z}{\mathbb{Z}}

% Keep cite/hyperref ordering (hyperref last)
\usepackage{cite}
\usepackage{hyperref}

\title{\textbf{Emergent Physical Laws from Self\,–\,Referential Quantum\,–\,Information Dynamics (SRQID):}\\\textbf{A Linear, Measurement\,–\,Free Hamiltonian with Lieb\,–\,Robinson Bounds}}

\author{Joshua Farrow\\\textit{Independent Research}\\\texttt{joshuajamesfarrow@protonmail.com}}
\date{\today}

\begin{document}
\maketitle

\begin{abstract}
We present \emph{\srqid{}}, a strictly \emph{linear}, \emph{time\,–\,independent}, and \emph{measurement\,–\,free} Hamiltonian framework in which matter qubits live on graph vertices and finite\,–\,dimensional bond registers live on edges. All updates are generated by a fixed, bounded\,–\,degree, local Hamiltonian; there is no state\,–\,dependent generator and no projective measurement. Locality implies a finite Lieb\,–\,Robinson (LR) velocity that defines an emergent light cone. We provide exact structural statements and minimal numerics: commutator\,–\,growth extraction of the LR velocity, a no\,–\,signalling local\,–\,quench test, and energy\,–\,drift bounds. A companion paper (QATNU) develops a circuit realization (QCA+promotion) and phenomenology; here we supply the linear foundation it cites \cite{QATNUcompanion}.
\emph{Code and minimal outputs:} Harvard Dataverse
(\href{https://dataverse.harvard.edu/dataset.xhtml?persistentId=doi:10.7910/DVN/YZE8RI}{doi:10.7910/DVN/YZE8RI}).
\end{abstract}

\section{Introduction}
A recurring challenge for emergent\,–\,geometry proposals is the use of non\,–\,linear or measurement\,–\,based updates. \srqid{} is built to avoid both: the generator is a \emph{fixed, local, time\,–\,independent} Hamiltonian on a bounded\,–\,degree graph; apparent ``selection'' or ``promotion'' phenomena arise from coherent dynamics in an \emph{enlarged Hilbert space} with isometric ladders on bonds, not from projections.

\paragraph{Degrees of freedom.} Let $G=(V,E)$ be a bounded\,–\,degree simple graph. Each $i\in V$ hosts a two\,–\,level matter system with Pauli operators $\{X_i,Y_i,Z_i\}$. Each edge $e=\langle i,j\rangle\in E$ hosts a finite\,–\,dimensional bond register with Pauli\,–\,like operators $\{\tau^{x,y,z}_{ij}\}$ acting on the lowest nontrivial subspace; higher levels are accessed via an isometric ladder (below).

\paragraph{Conventions.} We set $\hbar=1$ and use operator norms throughout.

\section{Hamiltonian model}
We consider the following time\,–\,independent Hamiltonian on $\hilbert=\bigotimes_{i\in V}\mathbb{C}^2\otimes\bigotimes_{\langle i,j\rangle\in E}\mathbb{C}^{\chi_{ij}}$:
\begin{equation}
\label{eq:H}
H\;=\; H_\text{mat} + H_\text{bond} + H_\text{int},
\end{equation}
with (one convenient choice)
\begin{align}
H_\text{mat} &= \frac{\omega}{2}\sum_{i\in V} X_i,\\
H_\text{bond} &= \sum_{\langle i,j\rangle\in E}\Big( J_0\, Z_i Z_j + \Delta_b\, n_{ij} \Big),\label{eq:Hbond-explicit}\\
H_\text{int} &= g\sum_{\langle i,j\rangle\in E} F_{ij}\,\tau^{x}_{ij} \; + \; \kappa\sum_{\langle i,j\rangle\in E} C_{ij} \;+\; \lambda\sum_{\langle i,j\rangle\in E} F_{ij}\otimes(R_{ij}+R_{ij}^\dagger),\label{eq:Hint-explicit}
\end{align}
where:
\begin{itemize}
\item $F_{ij}$ and $C_{ij}$ are \emph{fixed, local} operators (constructed from short Pauli strings) that (i) respond to frustration/entanglement locally and (ii) softly enforce bounded degree (no state dependence). A concrete tile is $F_{ij}=\tfrac{1}{2}(\id-Z_i Z_j)$ and $C_{ij}$ a diagonal penalty that \emph{disfavors} simultaneous activations beyond a chosen local bound.
\item $n_{ij}=\sum_{m\ge 0} m\,\Pi^{(m)}_{ij}$ is the rung (bond-occupancy) operator on edge $\langle i,j\rangle$ with spectral projectors $\{\Pi^{(m)}_{ij}\}_{m=0}^{\chi_{ij}-1}$.
\item $R_{ij}=\sum_{m=0}^{\chi_{ij}-2}\ket{m{+}1}\!\bra{m}$ is the local ladder-raising isometry on the bond register (finite truncation). The last term in \eqref{eq:Hint-explicit} provides an \emph{explicit Hamiltonian mechanism} for coherent promotion $m\!\to\! m\pm 1$ when the local frustration tile $F_{ij}$ is active. No projectors or state-conditioned generators are introduced; the enlarged Hilbert space and all couplings are present from $t=0$.
\end{itemize}

\paragraph{Local degree and active-rung operators.}
Define the operator degree at vertex $i$ by
\[
d_i := \sum_{j\in \partial i} \sum_{m\ge 1} \Pi^{(m)}_{ij},\qquad \partial i:=\{j:\langle i,j\rangle\in E\}.
\]
A convenient bounded penalty is $C_{ij}:= (d_i-k_0)^2+(d_j-k_0)^2$, which is diagonal and preserves linearity. We realize ``active-rung'' action by
\[
\tau^{\alpha}_{ij} := \sum_{m} \Pi^{(m)}_{ij}\,\tau^{\alpha,(m)}_{ij}\,\Pi^{(m)}_{ij},\qquad \alpha\in\{x,z\},
\]
so $\tau^\alpha_{ij}$ is block-diagonal and strictly local.

\paragraph{Isometric ladders \emph{vs.} measurements.} The Stinespring/Choi picture \cite{Stinespring1955,Choi1975} tells us any isometry can be realized unitarily on an enlarged space without projective measurement. Here the ladder is built directly into $H$ via $n_{ij}$ and $R_{ij}$; Eq.~\eqref{eq:H} is time independent and linear on $\hilbert$ at all times.

\paragraph{Entanglement proxy (for plots only).} For diagnostics we use the standard two\,–\,site purity proxy $\Delta_{ij}=1-\tr\rho_i^2-\tr\rho_j^2+\tr\rho_{ij}^2\in[0,1]$ computed \emph{from trajectories}. The Hamiltonian does not depend on $\Delta_{ij}$.

\paragraph{Clarification (penalties $\neq$ measurements).}
$C_{ij}$ denotes a Hamiltonian penalty (diagonal operator), not a measurement; the generator in Eq.~\eqref{eq:H} is state-independent and linear throughout.

\section{Locality and Lieb\,–\,Robinson bound}
\paragraph{Bounded-degree LR cone (standard).}
For finite-range, bounded-norm interactions on a bounded-degree graph, Lieb–Robinson (LR) bounds guarantee an emergent causal cone \cite{LiebRobinson1972,HastingsKoma2006,NachtergaeleSims2006}. Our $H$ satisfies these hypotheses by construction; thus there exist $v_{\mathrm{LR}},\xi,c>0$ such that for any $X,Y\subset V$ and local observables $A_X$, $B_Y$,
\begin{equation}
\label{eq:LR}
\big\|[A_X(t),B_Y]\big\|\;\le\; c\,\|A_X\|\,\|B_Y\|\,\exp\!\left(-\frac{\mathrm{dist}(X,Y)-v_{\mathrm{LR}}|t|}{\xi}\right).
\end{equation}

\paragraph{An explicit conservative constant.}
Let $H=\sum_{X\subset V} h_X$ with $\mathrm{diam}(X)\le R$ and define
\begin{equation}
J_* \,:=\, \sup_{x\in V}\,\sum_{X\ni x} \|h_X\|,\qquad z:=\sup_{x\in V}|\partial x| \text{ (max degree)}.
\end{equation}
Then for any $\mu>0$ one has an LR bound of the form \eqref{eq:LR} with velocity
\begin{equation}
\label{eq:vLR-explicit}
v_{\mathrm{LR}}(\mu)\;\le\; \frac{2}{\mu}\,C_\mu,\quad C_\mu:=\sup_{x\in V}\sum_{X\ni x}\|h_X\|\,|X|\,e^{\mu\,\mathrm{diam}(X)}\,\le\, J_*(1+z)\,e^{\mu R}.
\end{equation}
Optimizing over $\mu$ yields an explicit $v_{\mathrm{LR}}\lesssim 2e\,R\,J_*(1+z)$, which in our case (nearest-neighbor, $R{=}1$) depends only on $\omega, J_0,\Delta_b,g,\lambda,\kappa$, the operator norms $\|F\|,\|\tau^{x,z}\|$, and the degree cap $z$.\footnote{This is a standard Hastings--Koma/Nachtergaele--Sims constant; we state it here for parameter transparency. See also \cite{HastingsKoma2006,NachtergaeleSims2006}.}

\section{Exact structural statements}
\begin{itemize}
  \item \textbf{Linearity and measurement\,–\,free evolution:} The generator Eq.~\eqref{eq:H} is time\,–\,independent and does not condition on the quantum state.
  \item \textbf{Bounded degree (soft cap):} The $C_{ij}$ terms penalize configurations that would exceed a chosen local degree.
  \item \textbf{Isometry and unitarity:} The ladder is realized Hamiltonianly via $n_{ij}$ and $R_{ij}$; evolution remains unitary on the enlarged Hilbert space.
\end{itemize}

\section{Mathematical foundations: self-adjointness and growth bounds}
\paragraph{Self-adjointness and thermodynamic limit.}
\begin{lemma}[Essential self-adjointness and unitary dynamics]
For finite-range, bounded-norm interactions on a bounded-degree graph, $H$ defined by \eqref{eq:H}--\eqref{eq:Hint-explicit} is essentially self-adjoint on the local algebra. The dynamics $U(t)=e^{-itH}$ is a strongly continuous one-parameter unitary group on the quasi-local algebra, and satisfies an LR bound of the form \eqref{eq:LR}.
\end{lemma}
\begin{proof}[Proof sketch]
Decompose $H=\sum_X h_X$ with $\sup_X\|h_X\|<\infty$ and $\mathrm{diam}(X)\le R<\infty$. Standard constructions (e.g., \cite{NachtergaeleSims2006}) show the thermodynamic limit of local Hamiltonians exists, is essentially self-adjoint on the local observable algebra, and generates a strongly continuous dynamics obeying an LR bound.
\end{proof}

\paragraph{Degree growth bound and no runaway connectivity.}
\begin{lemma}[Local degree growth is LR-limited]
Let $d_i=\sum_{j\in\partial i}\sum_{m\ge1}\Pi^{(m)}_{ij}$. Then
\begin{equation}
\frac{d}{dt}\,\expect{d_i(t)} \;=\; i\,\expect{[H,d_i](t)} \quad\Rightarrow\quad \left|\frac{d}{dt}\,\expect{d_i(t)}\right| \;\le\; \|[H,d_i]\| \;\le\; \alpha,
\end{equation}
with $\alpha$ depending only on local operator norms and the degree cap $z$. Moreover, for any region $X$ and its complement at distance $r$, $\expect{d_i(t)}$ outside the LR cone differs from its unperturbed value by at most $O(e^{-(r-v_{\mathrm{LR}} t)/\xi})$.
\end{lemma}
\begin{proof}[Proof sketch]
Only terms in $H$ supported on edges incident to $i$ fail to commute with $d_i$, yielding a uniform bound $\|[H,d_i]\|\le c_0(\|F\|(|g|{+}|\lambda|)+|\kappa|\,\|C\|)$ with $c_0$ depending on $z$. The LR bound then controls the spread of any difference outside the cone.
\end{proof}

\paragraph{Promotion speed $\le$ LR speed.}
\begin{proposition}[LR-limited promotion]
For every edge $\langle i,j\rangle$ and rung projectors $\{\Pi^{(m)}_{ij}\}$, the expectation $\sum_m m\,\expect{\Pi^{(m)}_{ij}(t)}$ changes with a finite velocity bounded by the LR constants. In particular, a local quench supported in $X$ alters $\expect{\Pi^{(m)}_{ij}(t)}$ at distance $r$ by at most $O(e^{-(r-v_{\mathrm{LR}} t)/\xi})$.
\end{proposition}
\begin{proof}[Proof sketch]
Apply \eqref{eq:LR} to the commutator $[\Pi^{(m)}_{ij}(t),B_Y]$ with $B_Y$ supported in the quench region $Y=X$. The Duhamel expansion and Grönwall inequality give the stated exponential suppression outside the cone.
\end{proof}

\begin{remark}[Clifford equivalence of stabilizer tiles]
The concrete choice $F_{ij}=\tfrac12(\id-Z_iZ_j)$ is without loss: alternative tiles used in the QATNU companion (e.g., $Y_iY_j$, or short stabilizer strings) are related by local Clifford rotations, leaving locality and LR constants invariant (up to norm factors).
\end{remark}

\section{Validated Numerics (2025\,–\,08\,–\,12)}
Minimal, checkable numerics supporting the SRQID claims (details and scripts in the companion notebook; environment pinned in \texttt{requirements\_qatn.txt}).
\begin{itemize}
  \item \textbf{Lieb\,–\,Robinson velocity:} From commutator\,–\,growth thresholds (\(\varepsilon=10^{-3}\)) we extract $v_{\mathrm{LR}}\approx 2.071$ on $N=8$.
  \item \textbf{No\,–\,signalling local quench:} Flipping site $0$ at $t=0$ changes the far\,–\,end expectation by $\max_t\,|\Delta\langle Z_r(t)\rangle|\approx 3.44\times10^{-15}$ at $r=N{-}1$ (sub\,–\,cone propagation).
  \item \textbf{Energy conservation:} For the time\,–\,independent $H$, the energy drift is $\Delta E\approx 5.33\times10^{-15}$ across the evolution window.
  \item \textbf{Circuit cross\,–\,check (reported from QATNU):} a separate layer\,–\,unitarity test gives $\|U^{\dagger}U-I\|_2=1.83\times10^{-15}$.
\end{itemize}

\subsection*{Numerical parameters and integrator details}
All specific scalars used for the plots (\(\omega,J_0,\Delta_b,g,\lambda,\kappa,k_0\)), lattice size/boundaries, time window, and tolerance are \emph{emitted at run time} to \texttt{outputs/summary.txt} by \texttt{make run}. The commutator norms and quenches are computed by exact diagonalization (dense eigensolver, double precision); time evolution uses the spectral decomposition $e^{-itH}$, and reported energy drift is at machine precision. Re-running the default configuration in the Dataverse bundle reproduces the numbers above.

\section{Methods (numerical)}
\paragraph{Commutator growth.} We evaluate $\|[Z_0(t),Z_r]\|$ by exact diagonalization for $N\le 8$, record first\,–\,passage times $t_\text{hit}(r;\varepsilon)$ at threshold $\varepsilon$, and fit $r\approx v_{\mathrm{LR}}\,t + b$.

\paragraph{No\,–\,signalling quench.} Starting from $\ket{0}^{\otimes N}$ we compare $\langle Z_r(t)\rangle$ with and without an $X$ at site 0 at $t=0$; the difference remains below numerical noise until the cone arrival, consistent with \eqref{eq:LR}.

\paragraph{Energy drift.} We compute $E(t)=\expect{H(t)}$ along trajectories and report $\max_tE-\min_tE$.

\paragraph{Reproducibility.}
All scripts and the pinned environment file (\texttt{requirements\_qatn.txt})
are provided as arXiv ancillary files and archived on Harvard Dataverse
(\href{https://dataverse.harvard.edu/dataset.xhtml?persistentId=doi:10.7910/DVN/YZE8RI}{doi:10.7910/DVN/YZE8RI}).
Running \texttt{make run} reproduces the LR-velocity fit, quench test, and energy-drift report.

\section{Discussion}
\srqid{} supplies a measurement\,–\,free linear backbone with a controlled causal structure. It is intentionally conservative: we do not claim gauge cascades or spin\,–\,2 fits here. Those appear in QATNU and cite this paper for linearity and LR bounds \cite{QATNUcompanion}. Extensions include larger\,–\,$N$ tensor\,–\,network simulations, disorder robustness, and bounds on effective cone renormalization under ladder depth.

\section*{Data and code availability}
The reproducibility bundle (Makefile, pinned requirements, and scripts for SRQID and QATNU) is archived on
Harvard Dataverse at
\href{https://dataverse.harvard.edu/dataset.xhtml?persistentId=doi:10.7910/DVN/YZE8RI}{doi:10.7910/DVN/YZE8RI}.
A mirror of the scripts is included as arXiv ancillary files.

\section*{Acknowledgments}
The author thanks colleagues in quantum information, tensor networks, and quantum foundations for discussions and feedback.

\begin{thebibliography}{99}

\bibitem{LiebRobinson1972}
E. H. Lieb and D. W. Robinson,
``The finite group velocity of quantum spin systems,''
\textit{Commun. Math. Phys.} \textbf{28}, 251--257 (1972).

\bibitem{HastingsKoma2006}
M. B. Hastings and T. Koma,
``Spectral gap and exponential decay of correlations,''
\textit{Commun. Math. Phys.} \textbf{265}, 781--804 (2006).

\bibitem{NachtergaeleSims2006}
B. Nachtergaele and R. Sims,
``Lieb--Robinson bounds and the exponential clustering theorem,''
\textit{Commun. Math. Phys.} \textbf{265}, 119--130 (2006).

\bibitem{Hensen2015}
B. Hensen \emph{et al.},
``Loophole-free Bell inequality violation using electron spins separated by 1.3 km,''
\textit{Nature} \textbf{526}, 682--686 (2015).

\bibitem{Stinespring1955}
W. F. Stinespring,
``Positive functions on C$^\ast$-algebras,''
\textit{Proc. Amer. Math. Soc.} \textbf{6}, 211--216 (1955).

\bibitem{Choi1975}
M.-D. Choi,
``Completely positive linear maps on complex matrices,''
\textit{Linear Algebra Appl.} \textbf{10}, 285--290 (1975).

\bibitem{NielsenChuang2010}
M. A. Nielsen and I. L. Chuang,
\textit{Quantum Computation and Quantum Information: 10th Anniversary Edition}
(Cambridge University Press, 2010).

\bibitem{SchumacherWerner2004}
B. Schumacher and R. F. Werner,
``Reversible quantum cellular automata,''
\textit{arXiv:quant-ph/0405174} (2004).

\bibitem{Arrighi2019}
P. Arrighi,
``An overview of quantum cellular automata,''
\textit{Nat. Comput.} \textbf{18}, 885--899 (2019).

\bibitem{QATNUcompanion}
J. Farrow,
``Quantum Automaton Tensor Network Universe (QATNU): Measurement-Free Gauge Promotion and Emergent Geometry,''
preprint (2025).

\bibitem{FarrowDataverse2025}
J. Farrow, \emph{QATNU and SRQID dataset and makefile}, Harvard Dataverse (2025),
\href{https://dataverse.harvard.edu/dataset.xhtml?persistentId=doi:10.7910/DVN/YZE8RI}{doi:10.7910/DVN/YZE8RI}.

\end{thebibliography}

\end{document}
