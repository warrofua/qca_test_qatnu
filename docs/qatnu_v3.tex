% QATNU — arXiv Draft (Aug 12, 2025)
\documentclass[12pt,a4paper]{article}
\usepackage[utf8]{inputenc}
\usepackage[T1]{fontenc}
\usepackage{amsmath,amssymb,amsthm}
\usepackage{physics}
\usepackage{color}
\usepackage{hyperref}
\usepackage{cite}
\usepackage{booktabs}
\usepackage{siunitx}
\usepackage{geometry}
\newtheorem{lemma}{Lemma}
\geometry{margin=1in}

% Custom commands
\newcommand{\qatnu}{\textsc{qatnu}}
\newcommand{\qca}{\textsc{qca}}
\newcommand{\mpu}{\textsc{mpu}}
\newcommand{\hilbert}{\mathcal{H}}
\newcommand{\expect}[1]{\langle #1 \rangle}
\newcommand{\tensor}{\otimes}
\providecommand{\tr}{\mathrm{tr}} % avoid conflict with physics.sty

\title{\textbf{Quantum Automaton Tensor Network Universe (QATNU):}\\\textbf{Measurement-Free Gauge Promotion and Emergent Geometry}}
\author{Joshua Farrow\\\textit{Independent Researcher}\\\texttt{joshuajamesfarrow@protonmail.com}}
\date{August 13, 2025}

\begin{document}
\maketitle

\begin{abstract}
We present Quantum Automaton Tensor Network Universe (\emph{\qatnu}), a strictly unitary, measurement-free, and locality-preserving circuit framework in which geometry is encoded in finite-dimensional bond registers on graph edges, while matter occupies vertex qubits. A fixed finite-depth quantum cellular automaton (\qca) reproduces Dirac-like propagation in the infrared \cite{Strauch2006,NayakVishwanath2000,Bisio2015,Raynal2017} and is augmented by a measurement-free promotion unit (\mpu) that is unitarily controlled by local stabilizer eigenvalues \cite{Gottesman1997,Poulin2005}. When local frustration crosses a threshold, the corresponding bond register promotes (\(\chi:1\!\to\!2\!\to\!4\!\to\!\cdots\)), changing effective adjacency and inducing long-wavelength excitations consistent with a spin-2 sector (analyzed with standard TT projectors) \cite{MaggioreBook,WeinbergBook}. Repeated local promotions implement a gauge cascade (\(U(1)\!\to\!SU(2)\!\to\!SU(3)\)) via isometric extensions (Choi-rank~1) \cite{Stinespring1955,Choi1975,NielsenChuang} (no measurements), providing a constructive route from stabilizer logic to non-Abelian dynamics \cite{Zohar2016,Banuls2020,ChandrasekharanWiese1997}. Locality is enforced by bounded-degree graphs and finite-range layers, implying a finite Lieb--Robinson velocity and an emergent light cone \cite{LiebRobinson1972,HastingsKoma2006,NachtergaeleSims2005}. We outline falsifiable windows (collider, astrophysical, tabletop) \cite{CMS2020aQGC,Abdo2009,LVK_GRtests} and a simulation roadmap; a companion paper (SRQID) \cite{FarrowSRQID2025} gives a Hamiltonian variant consistent with these constraints. \textbf{QATNU \& SRQID dataset:} \url{https://dataverse.harvard.edu/dataset.xhtml?persistentId=doi:10.7910/DVN/YZE8RI}
\end{abstract}


\section{Introduction}
Background-free and pregeometric approaches often add non-unitary ingredients or state-dependent generators. In \qatnu, updates that alter ``geometry'' are themselves \emph{unitary}: the \mpu{} acts conditionally on \emph{local} stabilizers\cite{Gottesman1997,Poulin2005}, so no projectors appear in the dynamics. The engine is a reversible, causal \qca\ in the standard sense\cite{SchumacherWerner2004,Arrighi2019,GNVW2012}. Locality is guaranteed by finite depth and finite interaction range, which entail a Lieb--Robinson light cone\cite{LiebRobinson1972,HastingsKoma2006,NachtergaeleSims2005}. Paper~I (SRQID) develops the Hamiltonian formalism; this article (Paper~II) develops the circuit engine, the promotion mechanism, and falsifiable traces.

\medskip\noindent\textbf{Imported LR cone (from SRQID).} For local observables $A_X$ and $B_Y$ supported on disjoint vertex sets $X,Y$ of a bounded-degree graph, the SRQID Hamiltonian proves the standard Lieb--Robinson bound
\begin{equation}
\label{eq:LR}
\big\|[A_X(t),B_Y]\big\|\;\le\; c\,\|A_X\|\,\|B_Y\|\,\exp\!\left(-\frac{\mathrm{dist}(X,Y)-v_{\!\rm LR}|t|}{\xi}\right),
\end{equation}
with positive constants $(c,\xi,v_{\!\rm LR})$ depending only on local norms and interaction range (see \cite{LiebRobinson1972,HastingsKoma2006,NachtergaeleSims2005}). Our circuit construction respects the same locality assumptions (finite depth/range), so Eq.~\eqref{eq:LR} applies mutatis mutandis to the effective light cone of \qatnu.

\section{Engine: QCA with Measurement-Free Promotion}
\subsection{Fixed brickwork layer}
Let sites \(i=1,\dots,N\) host matter qubits and edges \(\langle i,j\rangle\) host bond registers of local dimension \(\chi_{ij}\in\{1,2,4,\dots\}\). The base step is a two-layer brickwork unitary \(U_0\) of on-site Hadamards and controlled-phase gates along edges. It yields Dirac-like propagation in the long-wavelength limit on regular subgraphs\cite{Strauch2006,Bisio2015,Raynal2017}. 

For the standard 1D Hadamard walk the single-particle spectrum obeys
\begin{equation}\label{eq:dispersion}
\cos\,\omega(k)=\frac{\cos k}{\sqrt{2}}\,,\qquad |k|\le \pi,
\end{equation}
so \(\omega(k)\approx \pm k/\sqrt{2}\) near \(k=0\). Thus the group velocity is bounded by \(v_\mathrm{max}=1/\sqrt{2}\), realizing a discrete light cone\cite{NayakVishwanath2000}.

\subsection{Stabilizers and the promotion unit}
Choose a commuting, Hermitian, local stabilizer set \(\{S_{ij}\}\) that diagnoses frustration on edge \(\langle i,j\rangle\)\cite{Gottesman1997}. For concreteness we use two-qubit tiles such as \(S_{ij}=Y_iY_j\). Note that choices like \(S_{ij}=Z_iX_iZ_jX_j\) used in companion texts are \emph{Clifford-equivalent} to \(Y_iY_j\); a local Clifford change of basis maps one to the other, so they define the same promotion logic. The promotion step is the finite-depth unitary
\begin{equation}
U_{\mathrm{prom}}=\prod_{\langle i,j\rangle} \mathrm{C}[S_{ij}=-1]\;\mathrm{INC}_{ij},
\end{equation}
where \(\mathrm{INC}_{ij}\) coherently increments the bond register \(\chi_{ij}\to 2\chi_{ij}\) by appending ancillas in \(|0\rangle\) and applying a Clifford+T network that realizes an isometric embedding (Choi-rank~1)\cite{Stinespring1955,Choi1975,NielsenChuang,Barenco1995}. The full step is \(U=U_{\mathrm{prom}}U_0\). No projective measurement is used.


\noindent\emph{Clarification (projectors vs. measurements).} Any ``projector'' $C_{ij}$ referenced in SRQID denotes a bounded-norm \emph{Hamiltonian penalty} term, not a measurement; the evolution in both SRQID and \qatnu\ remains unitary and linear.


\noindent\emph{Bounded-degree control.} We adopt a soft bounded-degree policy (phase-penalized deferral of promotions) to preserve a uniform causal cone, consistent with the assumptions underlying Eq.~\eqref{eq:LR}.

\begin{lemma}[Parallel safety on brickwork]
Let $\mathcal{E}_\ell$ be a brickwork layer of disjoint edges. Then the set
$\{\mathrm{C}[S_{ij}=-1]\;\mathrm{INC}_{ij}\}_{\langle i,j\rangle\in\mathcal{E}_\ell}$
pairwise commutes; hence $U_{\mathrm{prom}}$ is order-independent and constant depth.
\end{lemma}

\noindent\textit{Sketch.} For disjoint $\langle i,j\rangle$ and $\langle k,\ell\rangle$, the stabilizers
$S_{ij}$ and $S_{k\ell}$ act on disjoint supports and commute; their spectral projectors
commute as polynomials in commuting observables. The targets $\mathrm{INC}_{ij}$ and
$\mathrm{INC}_{k\ell}$ act on disjoint bond registers, so they commute. \qed

\subsection{Gauge promotion gadgets}
Patterns of promoted bonds support compiled gadgets that implement \(SU(2)\) and \(SU(3)\) actions on subspaces transforming as fundamental representations, akin to quantum link and digital-simulation constructions in lattice gauge theory\cite{ChandrasekharanWiese1997,Zohar2016,Banuls2020,Calajo2024}. Unitarity follows from the isometry construction; a stabilizer-parity invariant suppresses mixed-anomaly obstructions (details deferred to SRQID).

\paragraph{Concrete $SU(2)$ gadget (minimal).} Encode a logical qubit at vertex $i$ in the single-excitation subspace of two promoted
bonds $a\equiv(i,j)$ and $b\equiv(i,k)$:
\begin{align*}
\mathcal{H}_{\mathrm{enc}} &= \mathrm{span}\{\ket{10}_{ab}, \ket{01}_{ab}\},\\
X_L &= \ket{10}\!\bra{01} + \ket{01}\!\bra{10},\\
Z_L &= \ket{10}\!\bra{10} - \ket{01}\!\bra{01},\\
Y_L &= i\,\ket{01}\!\bra{10} - i\,\ket{10}\!\bra{01}.
\end{align*}
Apply the compiled rotation, conditioned on the same local parity used by $U_{\mathrm{prom}}$,
\[
U_{\mathrm{SU(2)}}(\theta,\mathbf n)=\exp\!\Big[-\tfrac{i\theta}{2}\,(n_x X_L+n_y Y_L+n_z Z_L)\Big],
\]
which acts as a bona fide $SU(2)$ on $\mathcal{H}_{\mathrm{enc}}$ while preserving global unitarity.

\section{Emergent Geometry and Spin-2 Sector}
We define a metric proxy from two-point functions of promoted bonds and fit long-wavelength fluctuations to Pauli--Fierz structure (using standard TT projectors)\cite{MaggioreBook,WeinbergBook}. Planned numerics: small-\(N\) exact simulations to verify the causal cone and promotion thresholds; adaptive tensor-network/MERA runs (capped bond dimension) to study area-law surfaces and dimensional flow\cite{Vidal2007,Swingle2012}.

\section{Validated Numerics (2025--08--12)}
Minimal, checkable numerics supporting the main claims (exact scripts provided in SI):
\begin{itemize}
  \item \textbf{Local unitarity and stability:} Circuit-layer check gives $\|U^{\dagger}U - I\|_2 = 1.83\times 10^{-15}$.
  \item \textbf{Causality (SRQID):} From commutator growth at threshold $\varepsilon=10^{-3}$ we extract a Lieb--Robinson velocity $v_{\mathrm{LR}}\approx 2.071$\cite{LiebRobinson1972,HastingsKoma2006}.
  \item \textbf{Back-reaction (engine toy run):} Pearson correlation between metric-proxy fluctuations and stress-energy, $\mathrm{corr}(\delta\chi, T)=0.7191$ (shuffle controls reported in SI).
  \item \textbf{Spin-2 sector:} On promoted runs, a log--log fit yields $P(k)\propto k^{-n}$ with $n\approx 2.000$; transverse--traceless residuals are small\cite{MaggioreBook}.
  \item \textbf{Dimensional flow (toy MERA):} $D_H^{\mathrm{UV}}=1.5644\to D_H^{(1)}=2.0000$ after one coarse-graining layer\cite{Vidal2007,Swingle2012}.
  \item \textbf{Spectral dimension (promoted graph):} On a synthetic bounded-degree promoted graph we find $D_s\approx 1.335$, defined via heat-kernel return probability\cite{BurioniCassi1996,Calcagni2014}.
\end{itemize}

\begin{table}[t]
\centering
\caption{Sanity-check numerics mirrored from SRQID alongside the circuit-layer unitarity test.}
\label{tab:sanity}
\sisetup{scientific-notation=true}
\begin{tabular}{@{}ll@{}}
\toprule
\textbf{Quantity} & \textbf{Value (source)} \\
\midrule
$v_{\rm LR}$ (commutator growth, $\varepsilon=10^{-3}$) & $\approx 2.071$ (SRQID) \\
No-signalling quench $\max_r|\Delta\langle Z_r\rangle|$ & $\approx 3.44\times10^{-15}$ (SRQID) \\
Energy drift $\Delta E$ & $\approx 5.33\times10^{-15}$ (SRQID) \\
Circuit unitarity $\|U^{\dagger}U-I\|_2$ & $1.83\times10^{-15}$ (this work) \\
\bottomrule
\end{tabular}
\end{table}

\section{Falsifiable Windows}
\textbf{Collider.} Soft-cluster excesses at low $p_T$ and small $\Delta R$; deviations in quartic vertices (e.g.\ $\gamma\gamma\!\to\!W^+W^-$)\cite{CMS2020aQGC}; narrow vector-meson-like bumps associated with promotion gadgets; soft-photon bursts from ``Clifford refresh.''\\
\textbf{Astrophysical.} Dimension-7-suppressed photon dispersion and time-of-flight limits (cf.\ GRB~090510)\cite{Abdo2009}; gravitational-wave dispersion/phase tests\cite{LVK_GRtests}.\\
\textbf{Tabletop.} Optomechanical phase-noise scalings in large-baseline interferometers consistent with coherent, unitary back-reaction\cite{Aspelmeyer2014}.

\section{Discussion and Outlook}
Paper~I (SRQID) provides the linear Hamiltonian, a precise entanglement proxy $\Delta_{ij}$, and Lieb--Robinson bounds\cite{LiebRobinson1972,HastingsKoma2006,NachtergaeleSims2005}. Paper~II (this work) gives the constructive circuit and testable consequences. Immediate tasks: (i) finalize stabilizer tiles and gadget compilation; (ii) produce numerical fits for the spin-2 sector; (iii) quantify EFT constraints from symmetry-protected Lorentz violation\cite{Zohar2016,Banuls2020}.

\paragraph{Acknowledgments} The author thanks colleagues in quantum information, tensor networks, and quantum foundations for discussions and feedback.

\bibliographystyle{unsrt}
\begin{thebibliography}{99}

\bibitem{Strauch2006}
F. W. Strauch, ``Relativistic quantum walks,'' \textit{Phys. Rev. A} \textbf{73}, 054302 (2006).

\bibitem{NayakVishwanath2000}
A. Nayak and A. Vishwanath, ``Quantum walk on the line,'' arXiv:quant-ph/0010117 (2000).

\bibitem{Bisio2015}
A. Bisio, G. M. D'Ariano, P. Perinotti, and A. Tosini, ``Quantum cellular automata and free quantum field theory,'' \textit{Found. Phys.} \textbf{45}, 1137--1152 (2015).

\bibitem{Raynal2017}
P. Raynal, ``Simple derivation of the Weyl and Dirac quantum cellular automata,'' \textit{Phys. Rev. A} \textbf{95}, 062344 (2017).

\bibitem{Gottesman1997}
D. Gottesman, \textit{Stabilizer Codes and Quantum Error Correction}, PhD thesis (Caltech, 1997); arXiv:quant-ph/9705052.

\bibitem{Poulin2005}
D. Poulin, ``Stabilizer formalism for operator quantum error correction,'' arXiv:quant-ph/0508131 (2005).

\bibitem{MaggioreBook}
M. Maggiore, \textit{Gravitational Waves, Vol. 1: Theory and Experiments} (Oxford Univ. Press, 2007).

\bibitem{WeinbergBook}
S. Weinberg, \textit{Gravitation and Cosmology} (Wiley, 1972).

\bibitem{Stinespring1955}
W. F. Stinespring, ``Positive functions on C$^\ast$-algebras,'' \textit{Proc. Amer. Math. Soc.} \textbf{6}, 211--216 (1955).

\bibitem{Choi1975}
M.-D. Choi, ``Completely positive linear maps on complex matrices,'' \textit{Linear Algebra Appl.} \textbf{10}(3), 285--290 (1975).

\bibitem{NielsenChuang}
M. A. Nielsen and I. L. Chuang, \textit{Quantum Computation and Quantum Information} (Cambridge Univ. Press, 10th Anniversary ed., 2010).

\bibitem{Barenco1995}
A. Barenco \textit{et al.}, ``Elementary gates for quantum computation,'' \textit{Phys. Rev. A} \textbf{52}, 3457--3467 (1995).

\bibitem{Zohar2016}
E. Zohar, J. I. Cirac, and B. Reznik, ``Quantum simulations of lattice gauge theories using ultracold atoms,'' \textit{Rep. Prog. Phys.} \textbf{79}, 014401 (2016).

\bibitem{Banuls2020}
M. C. Ba\~nuls \textit{et al.}, ``Simulating lattice gauge theories within quantum technologies,'' \textit{Eur. Phys. J. D} \textbf{74}, 165 (2020).

\bibitem{ChandrasekharanWiese1997}
S. Chandrasekharan and U.-J. Wiese, ``Quantum link models: A discrete approach to gauge theories,'' \textit{Nucl. Phys. B} \textbf{492}, 455--471 (1997).

\bibitem{LiebRobinson1972}
E. H. Lieb and D. W. Robinson, ``The finite group velocity of quantum spin systems,'' \textit{Commun. Math. Phys.} \textbf{28}, 251--257 (1972).

\bibitem{HastingsKoma2006}
M. B. Hastings and T. Koma, ``Spectral gap and exponential decay of correlations,'' \textit{Commun. Math. Phys.} \textbf{265}, 781--804 (2006).

\bibitem{NachtergaeleSims2005}
B. Nachtergaele and R. Sims, ``Lieb--Robinson bounds and the exponential clustering theorem,'' \textit{Commun. Math. Phys.} \textbf{265}, 119--130 (2006); arXiv:math-ph/0506030.

\bibitem{SchumacherWerner2004}
B. Schumacher and R. F. Werner, ``Reversible quantum cellular automata,'' arXiv:quant-ph/0405174 (2004).

\bibitem{Arrighi2019}
P. Arrighi, ``An overview of quantum cellular automata,'' \textit{Nat. Comput.} \textbf{18}, 885--899 (2019).

\bibitem{GNVW2012}
D. Gross, V. Nesme, H. Vogts, and R. F. Werner, ``Index theory of one-dimensional quantum walks and cellular automata,'' \textit{Commun. Math. Phys.} \textbf{310}, 419--454 (2012).

\bibitem{Vidal2007}
G. Vidal, ``Entanglement renormalization,'' \textit{Phys. Rev. Lett.} \textbf{99}, 220405 (2007).

\bibitem{Swingle2012}
B. Swingle, ``Entanglement renormalization and holography,'' \textit{Phys. Rev. D} \textbf{86}, 065007 (2012).

\bibitem{BurioniCassi1996}
R. Burioni and D. Cassi, ``Universal properties of spectral dimension,'' \textit{Phys. Rev. Lett.} \textbf{76}, 1091--1093 (1996).

\bibitem{Calcagni2014}
G. Calcagni, D. Oriti, and J. Th\"urigen, ``Spectral dimension of quantum geometries,'' \textit{J. Phys. A: Math. Theor.} \textbf{47}, 355402 (2014).

\bibitem{CMS2020aQGC}
G. Aad \textit{et al.} (ATLAS Collaboration), ``Observation of photon-induced $W^+W^-$ production in $pp$ collisions at $\sqrt{s}=13$~TeV using the ATLAS detector,'' \textit{Phys. Lett. B} \textbf{816}, 136190 (2021); arXiv:2010.04019.

\bibitem{Abdo2009}
A. A. Abdo \textit{et al.} (Fermi LAT/GBM Collaborations), ``A limit on the variation of the speed of light arising from quantum gravity effects,'' \textit{Nature} \textbf{462}, 331--334 (2009).

\bibitem{LVK_GRtests}
R. Abbott \textit{et al.} (LIGO--Virgo--KAGRA Collaborations), ``Tests of General Relativity with GWTC-3,'' arXiv:2112.06861 (2021).

\bibitem{Aspelmeyer2014}
M. Aspelmeyer, T. J. Kippenberg, and F. Marquardt, ``Cavity optomechanics,'' \textit{Rev. Mod. Phys.} \textbf{86}, 1391--1452 (2014).

\bibitem{Calajo2024}
G. Calaj\`o \textit{et al.}, ``Digital quantum simulation of a (1+1)D SU(2) lattice gauge theory,'' \textit{PRX Quantum} \textbf{5}, 040309 (2024).

\bibitem{FarrowSRQID2025}
J. Farrow, ``Self-Referential Quantum-Information Dynamics (SRQID),'' SSRN Working Paper No.~5390753 (2025). Available at \url{https://papers.ssrn.com/sol3/papers.cfm?abstract_id=5390753}.

\end{thebibliography}

\end{document}
